\documentclass[a4paper, conference]{IEEEtran/IEEEtran}
\usepackage[utf8]{inputenc}
% Some very useful LaTeX packages include:

% *** MISC UTILITY PACKAGES ***
%\usepackage{ifpdf}
% Heiko Oberdiek's ifpdf.sty is very useful if you need conditional
% compilation based on whether the output is pdf or dvi.
% usage:
% \ifpdf
%   % pdf code
% \else
%   % dvi code
% \fi
% The latest version of ifpdf.sty can be obtained from:
% http://www.ctan.org/pkg/ifpdf

% *** CITATION PACKAGES ***
\usepackage{cite}
% cite.sty was written by Donald Arseneau
% V1.6 and later of IEEEtran pre-defines the format of the cite.sty package
% \cite{} output to follow that of the IEEE. Loading the cite package will
% result in citation numbers being automatically sorted and properly
% "compressed/ranged". e.g., [1], [9], [2], [7], [5], [6] without using
% cite.sty will become [1], [2], [5]--[7], [9] using cite.sty. cite.sty's
% \cite will automatically add leading space, if needed. Use cite.sty's
% noadjust option (cite.sty V3.8 and later) if you want to turn this off
% such as if a citation ever needs to be enclosed in parenthesis.
% cite.sty is already installed on most LaTeX systems. Be sure and use
% version 5.0 (2009-03-20) and later if using hyperref.sty.
% The latest version can be obtained at:
% http://www.ctan.org/pkg/cite
% The documentation is contained in the cite.sty file itself.

% *** GRAPHICS RELATED PACKAGES ***
%
\ifCLASSINFOpdf
  \usepackage[pdftex]{graphicx}
  % declare the path(s) where your graphic files are
  % \graphicspath{{../pdf/}{../jpeg/}}
  % and their extensions so you won't have to specify these with
  % every instance of \includegraphics
  % \DeclareGraphicsExtensions{.pdf,.jpeg,.png}
\else
  \usepackage[dvips]{graphicx}
\fi

% *** MATH PACKAGES ***
%
%\usepackage{amsmath}
% A popular package from the American Mathematical Society that provides
% many useful and powerful commands for dealing with mathematics.
%
% Note that the amsmath package sets \interdisplaylinepenalty to 10000
% thus preventing page breaks from occurring within multiline equations. Use:
%\interdisplaylinepenalty=2500
% after loading amsmath to restore such page breaks as IEEEtran.cls normally
% does. amsmath.sty is already installed on most LaTeX systems. The latest
% version and documentation can be obtained at:
% http://www.ctan.org/pkg/amsmath

% *** SPECIALIZED LIST PACKAGES ***
%\usepackage{listings} %For code in appendix
%\lstset
%{%Formatting for code in appendix
%    language=Java,
%    basicstyle=\footnotesize,
%    numbers=left,
%    stepnumber=1,
%    showstringspaces=false,
%    tabsize=1,
%    breaklines=true,
%    breakatwhitespace=false,
%    xleftmargin=.25in,
%}

%\usepackage{algorithmic}
% algorithmic.sty was written by Peter Williams and Rogerio Brito.
% This package provides an algorithmic environment fo describing algorithms.
% You can use the algorithmic environment in-text or within a figure
% environment to provide for a floating algorithm. Do NOT use the algorithm
% floating environment provided by algorithm.sty (by the same authors) or
% algorithm2e.sty (by Christophe Fiorio) as the IEEE does not use dedicated
% algorithm float types and packages that provide these will not provide
% correct IEEE style captions. The latest version and documentation of
% algorithmic.sty can be obtained at:
% http://www.ctan.org/pkg/algorithms
% Also of interest may be the (relatively newer and more customizable)
% algorithmicx.sty package by Szasz Janos:
% http://www.ctan.org/pkg/algorithmicx

% *** ALIGNMENT PACKAGES ***
%
%\usepackage{array}
% Frank Mittelbach's and David Carlisle's array.sty patches and improves
% the standard LaTeX2e array and tabular environments to provide better
% appearance and additional user controls. As the default LaTeX2e table
% generation code is lacking to the point of almost being broken with
% respect to the quality of the end results, all users are strongly
% advised to use an enhanced (at the very least that provided by array.sty)
% set of table tools. array.sty is already installed on most systems. The
% latest version and documentation can be obtained at:
% http://www.ctan.org/pkg/array


% IEEEtran contains the IEEEeqnarray family of commands that can be used to
% generate multiline equations as well as matrices, tables, etc., of high
% quality.

% *** SUBFIGURE PACKAGES ***
%\ifCLASSOPTIONcompsoc
%  \usepackage[caption=false,font=normalsize,labelfont=sf,textfont=sf]{subfig}
%\else
%  \usepackage[caption=false,font=footnotesize]{subfig}
%\fi
% subfig.sty, written by Steven Douglas Cochran, is the modern replacement
% for subfigure.sty, the latter of which is no longer maintained and is
% incompatible with some LaTeX packages including fixltx2e. However,
% subfig.sty requires and automatically loads Axel Sommerfeldt's caption.sty
% which will override IEEEtran.cls' handling of captions and this will result
% in non-IEEE style figure/table captions. To prevent this problem, be sure
% and invoke subfig.sty's "caption=false" package option (available since
% subfig.sty version 1.3, 2005/06/28) as this is will preserve IEEEtran.cls
% handling of captions.
% Note that the Computer Society format requires a larger sans serif font
% than the serif footnote size font used in traditional IEEE formatting
% and thus the need to invoke different subfig.sty package options depending
% on whether compsoc mode has been enabled.
%
% The latest version and documentation of subfig.sty can be obtained at:
% http://www.ctan.org/pkg/subfig

% *** FLOAT PACKAGES ***
%
%\usepackage{fixltx2e}
% fixltx2e, the successor to the earlier fix2col.sty, was written by
% Frank Mittelbach and David Carlisle. This package corrects a few problems
% in the LaTeX2e kernel, the most notable of which is that in current
% LaTeX2e releases, the ordering of single and double column floats is not
% guaranteed to be preserved. Thus, an unpatched LaTeX2e can allow a
% single column figure to be placed prior to an earlier double column
% figure.
% Be aware that LaTeX2e kernels dated 2015 and later have fixltx2e.sty's
% corrections already built into the system in which case a warning will
% be issued if an attempt is made to load fixltx2e.sty as it is no longer
% needed.
% The latest version and documentation can be found at:
% http://www.ctan.org/pkg/fixltx2e


%\usepackage{stfloats}
% stfloats.sty was written by Sigitas Tolusis. This package gives LaTeX2e
% the ability to do double column floats at the bottom of the page as well
% as the top. (e.g., "\begin{figure*}[!b]" is not normally possible in
% LaTeX2e). It also provides a command:
%\fnbelowfloat
% to enable the placement of footnotes below bottom floats (the standard
% LaTeX2e kernel puts them above bottom floats). This is an invasive package
% which rewrites many portions of the LaTeX2e float routines. It may not work
% with other packages that modify the LaTeX2e float routines. The latest
% version and documentation can be obtained at:
% http://www.ctan.org/pkg/stfloats
% Do not use the stfloats baselinefloat ability as the IEEE does not allow
% \baselineskip to stretch. Authors submitting work to the IEEE should note
% that the IEEE rarely uses double column equations and that authors should try
% to avoid such use. Do not be tempted to use the cuted.sty or midfloat.sty
% packages (also by Sigitas Tolusis) as the IEEE does not format its papers in
% such ways.
% Do not attempt to use stfloats with fixltx2e as they are incompatible.
% Instead, use Morten Hogholm'a dblfloatfix which combines the features
% of both fixltx2e and stfloats:
%
% \usepackage{dblfloatfix}
% The latest version can be found at:
% http://www.ctan.org/pkg/dblfloatfix

% *** PDF, URL AND HYPERLINK PACKAGES ***
%
\usepackage{url}
% url.sty was written by Donald Arseneau. It provides better support for
% handling and breaking URLs. url.sty is already installed on most LaTeX
% systems. The latest version and documentation can be obtained at:
% http://www.ctan.org/pkg/url
% Basically, \url{my_url_here}.

% *** Do not adjust lengths that control margins, column widths, etc. ***
% *** Do not use packages that alter fonts (such as pslatex).         ***
% There should be no need to do such things with IEEEtran.cls V1.6 and later.
% (Unless specifically asked to do so by the journal or conference you plan
% to submit to, of course. )

% correct bad hyphenation here
\hyphenation{op-tical net-works semi-conduc-tor}

\begin{document}
%
% paper title
% Titles are generally capitalized except for words such as a, an, and, as,
% at, but, by, for, in, nor, of, on, or, the, to and up, which are usually
% not capitalized unless they are the first or last word of the title.
% Linebreaks \\ can be used within to get better formatting as desired.
% Do not put math or special symbols in the title.
\title{Security Concerns Regarding Internet of Things}

% author names and affiliations
% use a multiple column layout for up to three different
% affiliations
\author{\IEEEauthorblockN{Jonathan Grangien}
\IEEEauthorblockA{M.Sc. Computer Science in Media Technology\\
Linköping University\\
Email: jonathan.grangien@gmail.com}}

% make the title area
\maketitle

% As a general rule, do not put math, special symbols or citations
% in the abstract
\begin{abstract}
The Internet of Things is a concept that has exploded in popularity in recent years, and is becoming increasingly so. By the year 2020 there will exist an estimated 24 billion of these devices.\cite{meola} How concerned about security issues should we be in this reality? Recent events suggest that many important precautions are yet to be made.
\end{abstract}

% Note that keywords are not normally used for peerreview papers.
\begin{IEEEkeywords}
Internet of Things, security, privacy, DDoS.
\end{IEEEkeywords}

% For peer review papers, you can put extra information on the cover
% page as needed:
% \ifCLASSOPTIONpeerreview
% \begin{center} \bfseries EDICS Category: 3-BBND \end{center}
% \fi
%
% For peerreview papers, this IEEEtran command inserts a page break and
% creates the second title. It will be ignored for other modes.
\IEEEpeerreviewmaketitle

\section{Introduction}
Devices labeled under the rubric \textit{Internet of Things} (henceforth IoT) are rapidly becoming widespread in common homes.\cite{mulani} The term was first used in 1999 by Kevin Ashton, British technology pioneer, to describe a system in which physical objects connect via sensors to the internet. While Ashton needed the term to illustrate the use of RFID tags to automate tracking goods in supply chains, the term has since come to be used for various scenarious where internet connectivity or computing capability can be used for various every day items. 

\subsection{Examples of IoT devices}
Examples of IoT devices vary greatly. Simple ones include screens on otherwise normally occuring objects in a household, e.g.\ refridgerators. The device that outputs on the display could provide the users with information retrieved from online data sources, and could be connected via WiFi. Devices could measure things like water and electricity consumption, possibly comparing this to online resources, and provide the users with advice.

Devices not related to home automation exist as well. In retail, systems that track supplies are useful, and digital payment via RFID or NFC are becoming widespread. In contrast to \textit{smart homes} (home automation), \textit{smart cities} are on the rise as well with things like parking space tracking devices\cite{parking} and street light intensity regulation.  

The popularity of the Raspberry Pi has introduced an accessible way for technology enthusiasts to tinker with IoT devices on their own, as the Raspberry Pi hardware is well suited for IoT at reasonable prices.\cite{raspberryiot} In their article \textit{Raspberry Pi as a Sensor Web node for home automation}, V. Vujović and M. Maksimović makes a case that the Raspberry Pi is an inexpensive device that can be utilised for a wide range of home automation projects and research.\cite{raspberrysensor} Figure~\ref{fig:raspberry} illustrates numerous use cases of such.

Some of these devices can be considered to be hobby gadgets that are mainly technologically impressive but do not add much necessary value to everyday life, while some can come to potentially transform the world. While many automated home devices may be unnecessary, things like smart lighting regulation could come to make a significant difference for the enviromnent in the future.\cite{meola}

\subsection{Risk of Things}
Cool gadgets and day to day life changing features aside, a vast expansion of small computers connected to the internet provides more targets for cyber crime. If hackers gain access to large numbers of devices, they can be rigged to simultaneously send large amounts of requests to services with the intent of causing traffic overload, commonly reffered to as Distributed Denial of Service (DDoS) attacks.\cite{ddos} Security breaches may also pose direct threats to human lives, in case of interference with devices related to, say cars or pacemakers.\cite{windriver}

There is also the issue of privacy protection.

This paper will discuss security and privacy issues with IoT\@: What they are and entail, measures that can be taken against them, and how much is being done about them today.

\begin{figure}[!t]
\centering
\includegraphics[width=3.3in]{assets/raspberry.jpg}
\caption{Integrated home automation system and its benefits}
\label{fig:raspberry}
\end{figure}

\section{Background}
This paper divides security concerns into three different categories: Malicious attacks, safety, and privacy. However, as threats to safety through cyber crime is often a result of breach of privacy, safety and privacy is discussed in combination. 

\subsection{Attacks}
\label{sec:attacks}
\subsubsection{Probably The First IoT Botnet Incident}
In 2014, researchers from Proofpoint uncovered what they believe might be the first IoT based attack, evolving more than 750,000 malicious email communications coming from more than 100,000 individual devices on a global scale.\cite{proofpoint} Specifically, about than 25 percent of the volume of the offending emails were not sent from conventional computers or mobile devices, but from devices such as routers, multi-media centers, smart TVs, and at least one seemingly innocent (to its owners) refrigerator. The attacks targeted many enterprises and individuals worldwide.

Something that contributed to making the attack difficult to block was that with the large amount of devices, no more than 10 emails were needed to be sent from a single device, making the sources vastly spread and hard to identify.\cite{proofpoint} It is also reported that many devices did not require advanced methods of compromisation to get included in the attack, instead user misconfigurations and use of default passwords left many devices easily susceptible to botnet takeover. 

Proofpoint conclude their report by stating that while IoT devices hold great promise to daily life, they also hold promise for criminal activity as they are easy to compromise. This being because vendors and users take few steps to make devices secure, and compromises are hard to detect.\cite{proofpoint} It should be noted however that the Proofpoint article is about two and a half years old at the time of writing this paper. %article from early 2014

\subsubsection{DynDNS Incident}
DDoS attacks are one of the most common ways to maliciously cause online services to go offline. On October 21, 2016 the major DNS provider DynDNS, which provides service for many high traffic sites and services such as Twitter, PayPal, GitHub, Netflix, Amazon and Spotify, experienced large amounts of DDoS attacks causing many of these sites to become inaccessible to millions of people worldwide for several hours.\cite{dynstatement} While the majority of internet users may have been outraged mainly by the downtime of the social media service Twitter, the result of this attack could have been critical for many businesses that rely on services like PayPal and Amazon.

Flashpoint, a tech team that provides insights into cyber crime,\cite{flashpoint} published an analysis stating that a significant portion of the infrastructure causing the attack were botnet linked devices compromised by the malware Mirai.\cite{fpmirai} Mirai scans for and compromises IoT devices like routers, security cameras and DVRs, that are protected by factory default usernames and passwords. Compromised devices are then connected to botnets. Its source code has been leaked and is hosted publicly (uploader claims it is for academic purposes, source withheld), which also means that virtually anyone can access it and potentially learn to use it to quickly pull connected devices into harming botnets.

Common hardware maker factory default usernames and passwords are generally not very strong, and the fact that many devices are left with the default usernames and passwords installed makes it very easy for malware like Mirai to compromise them.\cite{whomakes}

DynDNS released a public statement after the attack was mitigated, confirming that the Mirai malware was one source of the attacks and that in total, tens of millions of IP addresses were involved in the offense.\cite{dynstatement}

\subsubsection{The Largest DDoS Attack Known to Date}
In September, 2016 the hosting company OVH was subject to what is believed to be the largest DDoS attack to date, with traffic of over one terabyte per second at its highest.\cite{scmagazine} The founder of OVH stated that the attack was conducted by a botnet of 145,607 CCTV cameras and DVRs, each IP address source responsible for traffic mostly ranging from 1 to 30 megabyte per second.

\subsection{Privacy and Security}
Home automation devices produce enormous amounts of private data that provide possible targets for criminals. Maliciously compromising such devices could not only have them used in attacks, but could also be used to virtually invade someone's home. When using commercially manufactured IoT devices there is also the issue of agreeing to various terms of service, something that many may not do carefully, and may unwillingly have private details shared or published. An example could be manufactured IoT devices for cars, if the user is not careful there could potentially be an implementation of location tracking in the device, sending said user's drive routes to a company.\cite{meola}

\section{Discussion}
\subsection{Security and Safety Measures}
The incidents outlined in this paper share the common case that the IoT devices affected were easy to compromise. What may be the most fatal and simultaneously the easiest flaw to fix is the fact that many IoT devices are put into use with their factory system usernames and passwords still in place. This means that cracking one device leaves little resistance for cracking the others. The measure to be taken would be to simply use different default passwords for different devices, and make them more secure. A company that does not do this may be lacking basic comptence and/or interest with regards to security in IT, and may be a company primarily focused on sales. The same could be said for devices that do not have https security implemented, but simply http. Basic encryption goes a long way towards keeping out middlehand interceptions and packet sniffing.

An end user could marginally improve the security on their own by having a strong firewall implemented and/or put the device behind a VPN service. If possible it is also a great idea to change the factory username and password for reasons mentioned above, but that may require technical knowledge that most users do not have, and the hardware makers might not always make it easily accessible. 

A good solution for the factory default username/password problem would be if the hardware makers would require the users to change the default usernames and passwords, making sure that every device has different login credentials, while providing an easily accessible way to do so. As of writing this paper, several hardware makers are already implementing it, such as Samsung, Hikvision and Panasonic.\cite{ipvm} If every hardware maker would have done this, attacks referred to in~\ref{sec:attacks} would have been of much smaller scope as a great number of the devices involved in the botnets were factory default configured ones targeted by the malware like Mirai.

% estimate risk of sharing privacy data
%http://ieeexplore.ieee.org/stamp/stamp.jsp?arnumber=6849186

\subsection{Security Guidance}

\section{Conclusions}
Security is very important.

%\begin{figure}[!t]
%\centering
%\includegraphics[width=2.5in]{myfigure}
% where an .eps filename suffix will be assumed under latex, 
% and a .pdf suffix will be assumed for pdflatex; or what has been declared
% via \DeclareGraphicsExtensions.
%\caption{Simulation results for the network.}
%\label{fig_sim}
%\end{figure}

% Note that the IEEE typically puts floats only at the top, even when this
% results in a large percentage of a column being occupied by floats.


% An example of a double column floating figure using two subfigures.
% (The subfig.sty package must be loaded for this to work.)
% The subfigure \label commands are set within each subfloat command,
% and the \label for the overall figure must come after \caption.
% \hfil is used as a separator to get equal spacing.
% Watch out that the combined width of all the subfigures on a 
% line do not exceed the text width or a line break will occur.
%
%\begin{figure*}[!t]
%\centering
%\subfloat[Case I]{\includegraphics[width=2.5in]{box}%
%\label{fig_first_case}}
%\hfil
%\subfloat[Case II]{\includegraphics[width=2.5in]{box}%
%\label{fig_second_case}}
%\caption{Simulation results for the network.}
%\label{fig_sim}
%\end{figure*}
%
% Note that often IEEE papers with subfigures do not employ subfigure
% captions (using the optional argument to \subfloat[]), but instead will
% reference/describe all of them (a), (b), etc., within the main caption.
% Be aware that for subfig.sty to generate the (a), (b), etc., subfigure
% labels, the optional argument to \subfloat must be present. If a
% subcaption is not desired, just leave its contents blank,
% e.g., \subfloat[].


% An example of a floating table. Note that, for IEEE style tables, the
% \caption command should come BEFORE the table and, given that table
% captions serve much like titles, are usually capitalized except for words
% such as a, an, and, as, at, but, by, for, in, nor, of, on, or, the, to
% and up, which are usually not capitalized unless they are the first or
% last word of the caption. Table text will default to \footnotesize as
% the IEEE normally uses this smaller font for tables.
% The \label must come after \caption as always.
%
%\begin{table}[!t]
%% increase table row spacing, adjust to taste
%\renewcommand{\arraystretch}{1.3}
% if using array.sty, it might be a good idea to tweak the value of
% \extrarowheight as needed to properly center the text within the cells
%\caption{An Example of a Table}
%\label{table_example}
%\centering
%% Some packages, such as MDW tools, offer better commands for making tables
%% than the plain LaTeX2e tabular which is used here.
%\begin{tabular}{|c||c|}
%\hline
%One & Two\\
%\hline
%Three & Four\\
%\hline
%\end{tabular}
%\end{table}


% Note that the IEEE does not put floats in the very first column
% - or typically anywhere on the first page for that matter. Also,
% in-text middle ("here") positioning is typically not used, but it
% is allowed and encouraged for Computer Society conferences (but
% not Computer Society journals). Most IEEE journals/conferences use
% top floats exclusively. 
% Note that, LaTeX2e, unlike IEEE journals/conferences, places
% footnotes above bottom floats. This can be corrected via the
% \fnbelowfloat command of the stfloats package.


% trigger a \newpage just before the given reference
% number - used to balance the columns on the last page
% adjust value as needed - may need to be readjusted if
% the document is modified later
%\IEEEtriggeratref{8}
% The "triggered" command can be changed if desired:
%\IEEEtriggercmd{\enlargethispage{-5in}}

% references section

% can use a bibliography generated by BibTeX as a .bbl file
% BibTeX documentation can be easily obtained at:
% http://mirror.ctan.org/biblio/bibtex/contrib/doc/
% The IEEEtran BibTeX style support page is at:
% http://www.michaelshell.org/tex/ieeetran/bibtex/
%\bibliographystyle{IEEEtran}
% argument is your BibTeX string definitions and bibliography database(s)
%\bibliography{IEEEabrv,../bib/paper}
%
% <OR> manually copy in the resultant .bbl file
% set second argument of \begin to the number of references
% (used to reserve space for the reference number labels box)
%\begin{thebibliography}{1}

%\begingroup
%\RaggedRight
\bibliographystyle{./bib/IEEEtran}
\bibliography{./bib/lab4}
%\endgroup
%\printbiography

% example
%\bibitem{gamma}
%E. Gamma and R. Helm and R. Johnson and  J. Vlissides, \emph{Design Patterns: Elements of Reusable Object-Oriented Software}, \hskip 1em plus
%  0.5em minus 0.4em\relax Addison-Wesley, 1995.
%
%\bibitem{mulani}
%T. Mulani and S. Pingle (2016). Internet of Things. International Research Journal of Multidisciplinary Studies, [online] 2(3), p. Available at: http://www.irjms.in/sites/irjms/index.php/files/article/view/270/256 [Accessed 7 Nov. 2016].
%
%\bibitem{windriver}
%Windriver.com. (2016). [online] Available at: http://www.windriver.com/whitepapers/security-in-the-internet-of-things/wr\_security-in-the-internet-of-things.pdf [Accessed 7 Nov. 2016].
%
%\bibitem{meola}
%Meola, A. (2016). How the Internet of Things will affect security \& privacy. [online] Business Insider. Available at: http://www.businessinsider.com/internet-of-things-security-privacy-2016-8?r=US\&IR=T\&IR=T [Accessed 7 Nov. 2016].
%
%\bibitem{raspberryiot}
%M. Maksimović and V. Vujović and N. Davidović and V. Milošević and B. Perišić (2014). Raspberry Pi as Internet of things hardware: performances and constraints. [online] design issues. Available at: https://www.researchgate.net/profile/Branko\_Perisic/publication/\newline272175660\_Raspberry\_Pi\_as\_Internet\_of\_Things\_hardware\_Performances\newline\_and\_Constraints/links/54de02e40cf23bf204397efe.pdf [Accessed 7 Nov. 2016].
%
%\end{thebibliography}
% that's all folks
\end{document}

